\documentclass[12pt,a4paper]{article}

% Text
\usepackage[english]{babel}

% Math
\usepackage{amsmath}
\usepackage{amsthm}
\usepackage{amssymb}

% Algorithms
\usepackage{algorithm}
\usepackage[noend]{algpseudocode}

\begin{document}
	\begin{algorithm}
		\textbf{Input:} Array $ A = [a_1, a_2, ..., a_n] $.
		\\\textbf{Result:} Sorted array $ A $
		
		\begin{algorithmic}
			\Procedure{Merge\_Sort}{$ A $}
				\If{$ |A| \leq 1 $}
					\Return
				\EndIf
				\State $ mid \gets \lfloor |A|/2 \rfloor $
				\State $ A_l \gets [A_1, A_2, ..., A_{mid}] $
				\State $ A_r \gets [A_{mid+1}, A_{mid+2}, ..., A_{|A|}] $
				\\
				\State $ Merge\_Sort(A_l) $
				\State $ Merge\_Sort(A_r) $
				\\
				\State $ Combine(A, A_l, A_r) $
				\State \Return
			\EndProcedure
		\end{algorithmic}
		~\\
		\textbf{Input:} Original array $ A $, sorted arrays $ L $ and $ R $ of $ A $ corresponding to left and right subarrays of $ A $
		\\\textbf{Result:} $ L $ and $ R $ combined into $ A $ to form a sorted array $ A $

		\begin{algorithmic}
			\Procedure{Combine}{$ A, L, R $}
				\State $ l \gets 0 $ \Comment Here, we use 0-based indexing
				\State $ r \gets 0 $
				\State $ i \gets 0 $ \Comment Index for array $ A $
				\\
				\While{$ l < |L| $ OR $ r < |R| $ }
					\If{$ (l < |L|) $ AND $ (r \geq |R| $ OR $ L[l] \leq R[r]) $}
						\State $ A[i] \gets L[l] $
						\State $ l \gets l + 1 $
					\Else
						\State $ A[i] \gets R[r] $
						\State $ r \gets r + 1 $
					\EndIf
					\State $ i \gets i + 1 $
				\EndWhile
				\State \Return
			\EndProcedure
		\end{algorithmic}
	\end{algorithm}

	\begin{algorithm}
		\textbf{Input:} Array $ A $, left and right indices $ l $ and $ r $ corresponding to the
		index bounds of the array that should be sorted
		\\\textbf{Output:} Sorted array $ A $ (in-place) [only sorts between $ l $ and $ r $ inclusive]
		\begin{algorithmic}
			\Procedure{Quick\_Sort}{$ A, l, r $}
				\If {$ l < r $}
					\State $ p \gets Partition(A, l, r) $
					\State $ Quick\_Sort(A, l, p - 1) $
					\State $ Quick\_Sort(A, p + 1, r) $
				\EndIf
			\EndProcedure
			\\\\
			\textbf{Input:} Array $ A $, left and right indices $ l $ and $ r $ corresponding to bounds of the array to partition
			\\\textbf{Output:} Partitioned array $ A $ (in-place); The function itself returns the index $ p $ corresponding to the element $ e $ that was chosen for partitioning. Everything to the left of $ e $ will be smaller than $ e $, and everything to the right of $ e $ will be greater or equal to $ e $.
			\\
			\\\textbf{Lomuto partition scheme}
			\Procedure{Partition}{$ A, l, r $}
				\State $ pivot \gets A[r] $
				\State $ i \gets l $
				\\
				\For{$ j \gets l $ \textbf{to} $ r - 1 $}
					\If{$ A[j] < pivot $}
						\State Swap $ A[i] $ with $ A[j] $
						\State $ i \gets i + 1 $
					\EndIf
				\EndFor
				\\
				\State Swap $ A[i] $ with $ A[r] $
				\State \Return $ i $
			\EndProcedure
			\\
			\\\textbf{Hoare partition scheme}
			\Procedure{Partition}{$ A, l, r $}
				\State $ mid \gets l + (r - l)/2 $
				\State $ pivot \gets A[mid] $
				\State $ i \gets l - 1 $
				\State $ j \gets r + 1 $
				\While{$ true $}
					\State $ i \gets i + 1 $
					
					\While{$ A[i] < pivot $}
						\State $ i \gets i + 1 $
					\EndWhile
					\\
					\State $ j \gets j - 1 $
					
					\While{$ A[j] > pivot$}
						\State $ j \gets j - 1 $
					\EndWhile
					\\
					\If{$ i < j $}
						\State Swap $ A[i] $ with $ A[j] $
					\Else{}
						\State \Return $ j $
					\EndIf
				\EndWhile
			\EndProcedure
		\end{algorithmic}
	\end{algorithm}
\end{document}